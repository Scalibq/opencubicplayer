% OpenCP Module Player
%
% Documentation LaTeX source
%
% revision history: (please note changes here)
% -doj990413  Dirk Jagdmann  <doj@cubic.org>
%   -initial release

\chapter{Support}
\label{homepage}
The official internet homepage of \cp\ can be found under:
\begin{center}
\textbf{\htmladdnormallink{http://www.cubic.org/player}{http://www.cubic.org/player}}
\end{center}
If you have no access to the internet we have our bulletin board called
\emph{Digital Nightmare} located in Hildesheim/Germany. It can be reached via
the following numbers:
\begin{center}
\begin{tabular}{rl}
usr v.everything & +49-5121-39236 \\
v90 + isdn & +49-5121-102712 \\
isdn hispeed & +49-5121-157107 \\
\end{tabular}
\end{center}

Please send any suggestions and bug reports via electronic mail to Tammo
Hinrichs \textbf{\htmladdnormallink{opencp@gmx.net}{mailto:opencp@gmx.net}} and comments about
this documentation to Dirk Jagdmann \textbf{\htmladdnormallink{doj@cubic.org}{mailto:doj@cubic.org}}.

If you encountered a bug in \cp\ please send
a detailed bug report to Tammo Hinrichs. Many people complain about music
files not being played properly (especially with module formats). Please keep
in mind that the original tracking/sequencing program is \emph{always} the
reference for correct playback. If a music file sound different compared to
another player, please keep in mind that perhaps the other program does not
play it correct. This is especially true for Amiga 4channel modules. As those
files can be produced using various trackers (Noise-, Sound-, Protracker and
many, many more) and sadly all of those trackers play slightly different, a
general player like \cp\ can not play \emph{all} modules correctly. We try to
emulate the behaviour of ProTracker 1.1b, so that most modules are
played correctly. However certain features of one tracker permit another
feature of a different player.

\section{The Team}
Currently the core team of \cp\ devolopers consists of 4 people:

\begin{tabular}{r@{\quad-\quad}l}
Tammo Hinrichs & main coder \& ruler \\
Felix Domke & coder \\
Fabian Giesen & coder \\
Dirk Jagdmann & generic support, this and that \\
\end{tabular}
