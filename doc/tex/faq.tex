% OpenCP Module Player
%
% Documentation LaTeX source
%
% revision history: (please note changes here)
%  -doj990421  Dirk Jagdmann  <doj@cubic.org>
%    -initial release
%  -ryg990509  Fabian Giesen  <fabian@jdcs.su.nw.schule.de>
%    -beautified the faq a bit
%  -doj990627  Dirk Jagdmann  <doj@cubic.org>
%    -added entry about WMA/ASF
%  -doj990714  Dirk Jagdmann  <doj@cubic.org>
%    -moved the faq definitions into ocpmanual.cls and htmlmanual.cls
%  -doj20010724 Dirk Jagdmann <doj@cubic.org>
%    -added section about windows vapc

\chapter{Frequently Asked Questions}
this chapter was initially written by Mom/Cubic.

\faq{What is \cp?}
{\cp\ is a music player which plays a variety of sound formats on several sound cards.}

\faq{What sound cards are supported ?}
{Gravis UltraSound / MAX / DaughterBoard / PnP\\
SoundBlaster 1.x / 2.x / Pro / 16 / SB 32 / PnP / AWE / 64\\
WSS compatible cards\\
Pro Audio Spectrum series\\
ESS AudioDrive 688\\
Disk Writer, writes .WAV output on disk.\\
MPx Writer, writes .MP1 or .MP2 files on disk.\\
Quiet Player\\
Terratec EWS64L/XL}

\faq{What music formats are supported?}
{MOD/NST/WOW, XM, S3M, DMF, MTM, ULT, 669, OKT, PTM, AMS, MDL, IT, MIDI, SID, and MPEG 1 audio layer 1/2/3}

\faq{Which is the last version of \cp?}
{The last \cp\ version is the \cpversion\ Version.}

\faq{Can not find \texttt{dos4gw.exe}?}
{You should look at www.cubic.org/player for the program, we did not in clude it in the archive, as most DOS fanatics out there already have it.}

\faq{Where has \texttt{cplaunch.exe} gone?}
{It was killed, as it sucked. It caused severe timing errors with Wave table
cards, as \texttt{pmode/w} seems to have some problems with Interrupts.
Anyway, if you are in SUCH a need of it, you'll surely find someone who stil
has it}

\faq{Are you going to release new versions?}
{Perhaps not. We are all busy doing other things.}

\faq{Does \cp\ run with Windows 3.11?}
{\textbf{not perfect!} If you try it, your system might crash !}

\faq{The sound stops/hangs  when I try to switch \cp\ to background in Win9x -
how can I avoid it?}
{If you have a GUS or Interwave and use Hardware mixing, try to enable
the soundcard's timer (RTFM please) - if you use software mixing, note
that background playing is only available at the main player screen
for some technical reasons. If it doesn't work THEN - well, go complain
to your soundcard's driver programmer.}

\faq{Always when I'm in "X" mode (132 columns) and I try to switch back to
a window, \cp\ crashes! Why?}
{It's not \cp\ which crashes, it's Windows. Sadly, most graphic card
drivers get confused when you try to switch a 132 column mode to a
window - and hang the whole system. There's nothing that can be done
about it, except that you don't try this again. Sorry.}

\faq{I have an SB16 and use \cp\ under windows - why does the playing stop
whenever I shell to DOS and try to get back?}
{Actually, we have no idea. It seems that the SB16 Windows drivers modify
some of the card's registers in this moment which confuses either the
playing or the DOS shell port monitoring routines of \cp. }

\faq{Now, how does that MIDI player thing work?}
{\cp's MIDI player does not, like other programs, use the MIDI capa-
bilities of your sound card / Korg Trinity or whatever, but uses the
patch files of the famous Gravis Ultrasound to synthesize the MIDI
output itself (just like Timidity or kmidi under KDE) via it's normal
wavetable system. So the only thing you need are the GUS patch files
which are available at our site. Simply open a new subdir called "MIDI"
in \cp's directory, copy all patch files \texttt{.PAT} into it and there you
go.

Users of a Gravis Ultrasound only have to make sure their \texttt{ultradir}
environment variable is set correctly or state the patch file and
\texttt{ultrasnd.ini} location in \texttt{cp.ini}}

\faq{Does that mean I can listen to MIDI files even with my old SBPro?}
{Of course - if your soundcard works with the rest of \cp, you will be
able to play MIDI files, too - as long you as got the GUS Patch files as
written above.}

\faq{Hey, I have a module which \cp\ plays wrong! You SUCK!}
{Well, it's in fact YOU who sucks, as you didn't send that module together
with a bug report to opencp@gmx.net. How the hell can we fix bugs if
we don't know them?}

\faq{\cp\ plays my \texttt{.it} files much too soft, IT's internal mixer sounds a
lot better, why don't you just fix it?}
{Because this is a religious question. \cp's mixing routines are
designed for maximum sound quality which also includes things like declicking and quadratic interpolation, IT's mixing routines are just
optimized for speed without really caring about sound quality (speaking
of the non-MMX routines). So \cp's sound differs a bit, if you are
not pleased with it, feel free to use IT again ;)}

\faq{Can you give me the source or a part of the source from \cp?}
{The complete source code can be found on the official \cp\
sites, covered under the GNU GPL.}

\faq{I have a problem finding a small routine that will play mod files.
I wanted to know if you can help finding a routine, lib or something
like this.}
{Try our Tiny XM Player, a free XM Player inc. source.}

\faq{Why don't you implement Nibbles, or similar game, in Cubic? Like
FT2-nibbles? Or just plain Snake...}
{Get ahold of a coder, give him the source and force him to implement
it, if it's a GOOD nibbles version, we might include it into the
"official" release ;)}

\faq{I have a AWE32 and a GUS in my computer. Will both cards be supported
simultaneously in the nearby future?}
{Right now, only 2 GUSs' are supported simultaneously.}

\faq{What does ''W\"urfel Mode'' actually do?}
{''W\"urfel'' is german word for cube/dice. The mode plays an animation.}

\faq{Can I make my own ''W\"urfel Mode'' animation?}
{Of course you can. On our site you'll find the program WAP
(W\"urfelAnimator Professional) which can convert \texttt{.pcx} frames to a
W\"urfel Mode animation.}

\faq{Is it possible that you port \cp\ to other software platforms like
OS/2 and some free Unices (Linux and FreeBSD)?}
{No. The code is totally dependant on DOS and Watcom C. It would be
far easier to rewrite everything than porting. But we are not going to
rewrite anything at the moment.}

\faq{How can I put a picture to the background on \cp?}
{Convert your Picture to a TGA-file, these pics have to be 640x384x208 TGAs,
the first or the last 48 colors must not be used.
You can also save your image as a 640x384x208 gif87 picture. Either the
first or the last 48 colors of the palette have to be black.}

\faq{Up to version 2.0alpha++, \cp\ ran fine on my 486, now it creeps and can
only mix a hand full of channels. Can I fix this somehow?}
{RTFM ;)  No, as we're living in the age of Pentium computers now, we set
the Floating Point Mixer as default. Just change this.}

\faq{Why does this WMA-Player won't work?}
{You need the original MSAUD32.ACM in your path (or at the specified
location in the cp.ini). I only tested build 3688 and 3752.
Anyway, not all ASF-files are read correctly. All ASFs with a
blocksize of <256 won't work. Streams containing something else than
MSAudio (Codec 160) won't work.
Streams created with the WMT-Encoder and the WMAudio-SDK should work.

To use the WMA-Player under DOS, you have to use PMODE/W (and with this,
cplaunch.exe). Sorry for this, but this seems to be a bug in DOS4GW
(since it works in Win9x, DOSEMU (at least the decoder) and PMODE/W).}

\section{Windows Vapc driver specific}
\label{faqvapc}
The following are errors messages that might appear in the cphost log
along with a short description what caused them and how to avoid them.

\begin{description}

\item[The DLL \texttt{vocp.dll} couldn't be loaded]
the \texttt{vocp.dll} wasn't found. be sure that it's loadable, that
means somewhere in your path. \texttt{windows\symbol{92}system} should
be a good place

\item[WARNING: Couldn't set priority class! \emph{or} WARNING: Couldn't set thread priority!]
the host couldn't instruct windows to give him more cpu-power. there
will be serious skipping while playing.

\item[ERROR: Couldn't load \texttt{VAPC.VXD}! Is the \texttt{VAPC.VXD} in your texttt{windows\symbol{92}system}-directory?]
yeah, the \texttt{vapc.vxd} must be in your system directory.

\item[DEVPDX5: couldn't lock primary ... \emph{or} DEVPDX5: couldn't play on primary ...]
no problem, the driver will use secondary buffer.

\item[DEVPDX5: couldn't create directsound-object]
your soundcard is in use by some other stuff, maybe a dosbox, maybe
\emph{this} dosbox. remove all other hardware-devices from your
\emph{cp.ini}, as described in the beginning of this document.

\end{description}

If you own a sound card supported by \cp\ and have difficulties using
the vapc drivers there is the chance that both drivers interfere. You
may try to remove all native \cp\ drivers from the ini file. The
corresponing lines should look like this:

\begin{verbatim}
playerdevices=devpVXD devpNone devpDisk devpMPx
samplerdevices=devsNone
wavetabledevices=devwMixF devwmixQ devwMix devwNone
\end{verbatim}

\small{
   The native-driver will use the soundcard, and windows will reserve
   the device for the \cp-task.\footnote{better: \cp-VM} If this
   occurs, you will find something like \emph{couldn't create
   dsound-object} in the cphost-log. so, remove the other
   device-driver, restart the dosbox, and enjoy.

   It \emph{must} be possible to launch some direct-sound-output while
   the cp is \emph{running} (not neccesarily \emph{playing}). try a
   winamp or something first (using DirectSound, \emph{not}
   WaveOut). If this works, continue using \cp\ with DirectSound.
}
