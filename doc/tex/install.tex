% OpenCP Module Player
%
% Documentation LaTeX source
%
% revision history: (please note changes here)
% -doj990413  Dirk Jagdmann  <doj@cubic.org>
%   -initial release
% -doj20010724 Dirk Jagdmann <doj@cubic.org>
%   -added windows cphost.exe

\chapter{Installation and files}
\cp\ needs no special install procedure. Simply copy all files you found inside
the archive into one directory. If you like you can add this directory to your
path statement. Example:
\begin{quote}
Install \cp\ into \texttt{C:\symbol{92}OPENCP} by typing:
\begin{enumerate}
\item \texttt{C:}
\item \texttt{CD \symbol{92}}
\item \texttt{MD OPENCP}
\item \texttt{CD OPENCP}
\item \texttt{PKUNZIP -D C:[theRightDirectory]OPENCP}
\end{enumerate}
Then you can modify your \texttt{autoexec.bat} and add the new directory to
the path statement. The new path might look like \\
\texttt{path c:\symbol{92}windows;c:\symbol{92}windows\symbol{92}command;c:\symbol{92}opencp}.
\end{quote}

The following files are (at least) required to start \cp\ 
\begin{dojlist}
\item[cp.exe] the startup file used to start the \cp\ runtime system 
\item[cp.ini] the initializition file read by cp.exe to configure itself 
\item[cp.pak] the DLL library where all modules are stored 
\end{dojlist}

As \cp\ is a 32bit program it normally needs the DOS4GW extender to use
protected mode. If the file \texttt{dos4gw.exe} could not be found in the
current directory and the path, you get an error message.

When \cp\ loads a module the information stored inside the file is read and
stored in a special file (the \emph{module information cache}). Actually the
cache splits up into three files: 
\begin{dojlist}
\item[cparcs.dat] Information about archives and files stored inside 
\item[cpmdztag.dat] Paths to \texttt{.MDZ}-files (see appendix \ref{mdz})
\item[cpmodnfo.dat] Informations about various modules 
\end{dojlist}
If you delete these files all file information gathered is permanently lost.

The \texttt{doc} directory contains documentation files in various formats
and release notes etc. You can safely delete this directory if you are running
out of disk space.

\section{\cp\ and Windows}
\cp\ will run happily in a DOS-Box of Windows 9.x-ME. You can use the
native drivers of \cp, but this is not recommended with windows.
If you have a sound card supported by windows and installed a DirectX 5 or later
you can use the windows drivers with \cp. This way \cp\ will work with
every sound card supported by windows, which includes modern pci devices.

\begin{quote}
Installation is totally simple:
\begin{enumerate}
\item copy the \texttt{vxdapc.vxd} driver and the \texttt{vocp.dll} library into your texttt{windows\symbol{92}system} directory
\item run the \texttt{cphost.exe} program before starting \cp\
\end{enumerate}
\end{quote}

\texttt{cphost.exe} will place an icon in your icon tray to indicate proper operation. If you use \cp\ regularily you can start \texttt{cphost.exe} with your autostart group.\footnote{Either move it into the autostart folder, or place a link in there.}

If a yellow exclamation sign is added to the icon (in your icon tray), something failed. Please contact us for a bug report together with system information, the log from \texttt{cphost.exe} etc.

All this however only applies to Windows 9.x-ME. \cp\ does not run
with Windows NT, 2K or XP. If you have problems please see our section
about the vapc driver in the FAQ.\footnote{page \pageref{faqvapc}}

\section{Notes for german users}
Auf der deutschen Tastatur weichen die Tastenbezeichnungen bei den Sonderzeichen
von denen der englischen Tastatur ab. In dieser Anleitung wurden dabei immer die
englischen Bezeichnungen verwendet. Mit der folgenden Tabelle sollten aber alle
Unklarheiten beseitigt sein.\footnote{Verwechseln Sie nicht die
\keys{$\longleftarrow$}\ (Backspace) mit der \keys{$\leftarrow$}-Taste! Die
\keys{$\longleftarrow$}-Taste liegt normalerweise \"uber der
\keys{$\hookleftarrow$}-Taste (Enter/Return).}

\begin{tabular}{r@{ -- }l}
english & deutsch \\ \hline
\keys{SHIFT} & \keys{$\Uparrow$} \\
\keys{CTRL} & \keys{STRG} \\
\keys{Ins} & \keys{Einfg} \\
\keys{Del} & \keys{Entf} \\
\keys{Home} & \keys{Pos1} \\
\keys{End} & \keys{Ende} \\
\keys{Pgup} & \keys{Bild$\uparrow$} \\
\keys{Pgdown} & \keys{Bild$\downarrow$} \\
\keys{Backspace} & \keys{$\longleftarrow$} \\
\keys{Enter} & \keys{$\hookleftarrow$} \\
\keys{Space} & \keys{Leertaste} \\
\keys{Tab} & \keys{$|\leftarrow$}
\end{tabular}
